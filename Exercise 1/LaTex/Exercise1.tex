\documentclass[11pt, a4paper]{article}	% Alternativet er å bruke {report}, men vi holder oss helst til {article} i labrapportene
\usepackage[utf8]{inputenc}
\usepackage[T1]{fontenc} 								% Vise norske tegn

\usepackage{graphicx} 
      						% For å inkludere figurer
%\setcounter{secnumdepth}{0} 						% \section nummereres ikke

% Justeringer av defaultverdier for teksten på siden:
\setlength{\textheight}{240mm} 
\setlength{\textwidth}{150mm}  
\topmargin -5mm 
\oddsidemargin -5mm


%%%%%%%%%%%%%%%%%%%%%%%%%%%%%%%%%%%%%%%%%%%%%%%%%%%%%%%%%%%%%%%%%%%%%%%%%
\begin{document}

\begin{titlepage}
\newcommand{\HRule}{\rule{\linewidth}{0.5mm}}
 
\begin{center}
 

\textsc{\huge TDT4190 }\\[2.0cm]

 
\HRule \\[0.3cm]
{\huge \bfseries  Øving  1
\\\ \\\ \LARGE Kjetil Sletten, Simen Skoglund og Christian Peter }\\[0.4cm]
\HRule \\[1.5cm]
 
\vfill
 
{\large \today}
 
\end{center}
\end{titlepage}

\section{Kort beskrivelse}
Vi har valgt å lage en egen klasse som implementer Java RMI. Hver klasse er beskrevet nedenfor.

\section{Connector.java}

\section{ConnectorInterface.java}
Grunnen til at vi har med denne er at vi trenger et interface som kan brukes av klienten på den andre siden. Dette grensesnittet implementerer 

\section{BoardModel.java}
\subsection*{public void cleanBoard}
Denne metoden måtte jeg legge til for å fjerne alle kryss og sirkler. Ingenting som har med RMI er gjort her.

\section{TicTacToe.java}


\newpage
\begin{thebibliography}{2}


\end{thebibliography}

\end{document}